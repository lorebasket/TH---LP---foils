
% how to change Contents to Table of Contents
\addto\captionsenglish{% Replace "english" with the language you use
  \renewcommand{\contentsname}%
    {Table of Contents}%
}

% to change the name of Abbreviations to Acronyms
% not needed if use use entry types and define those
% \renewcommand{\abbreviationsname}{Acronyms}

% to allow line comments in algorithms
\algnewcommand{\LineComment}[1]{\State \(\triangleright\) #1}

% to declare abs and norm
\DeclarePairedDelimiter\abs{\lvert}{\rvert}%
\DeclarePairedDelimiter\norm{\lVert}{\rVert}%

% Swap the definition of \abs* and \norm*, so that \abs
% and \norm resizes the size of the brackets, and the 
% starred version does not.
\makeatletter
\let\oldabs\abs
\def\abs{\@ifstar{\oldabs}{\oldabs*}}
%
\let\oldnorm\norm
\def\norm{\@ifstar{\oldnorm}{\oldnorm*}}
\makeatother


\lstdefinestyle{CStyle}{
    backgroundcolor=\color{backgroundColour},   
    commentstyle=\color{mGreen},
    keywordstyle=\color{magenta},
    numberstyle=\tiny\color{mGray},
    stringstyle=\color{mPurple},
    basicstyle=\scriptsize\ttfamily,
    breakatwhitespace=false,         
    breaklines=true,                 
    captionpos=b,                    
    keepspaces=true,                 
    numbers=left,                    
    numbersep=5pt,                  
    showspaces=false,                
    showstringspaces=false,
    showtabs=false,                  
    tabsize=2,
    language=C,
    lineskip=-0.1ex
}

\newcommand{\Dedications}{%
    \chapter*{Dedications}
    \addcontentsline{toc}{chapter}{Dedications} % Aggiunge 'Dedications' all'indice
}

\newcommand{\paginavuota}{
    \newpage
    \thispagestyle{empty}
}

% Formato per subsubsection
\titleformat{\subsubsection}[block]
  {\normalfont\normalsize\bfseries} % Stile del titolo
  {\thesubsubsection} % Aggiungi il numero di subsubsection
  {1em} % Spaziatura tra numero e titolo
  {} % Comando dopo il titolo

\renewcommand\thesubsubsection{\thesection.\arabic{subsection}.\arabic{subsubsection}} % Formato della numerazione per subsubsection

% Formato per paragraph
\titleformat{\paragraph}[runin]
  {\normalfont\normalsize\bfseries} % Stile del titolo
  {\theparagraph} % Aggiungi il numero di paragraph
  {1em} % Spaziatura tra numero e titolo
  {} % Comando dopo il titolo

\renewcommand\theparagraph{\thesubsubsection.\arabic{paragraph}} % Formato della numerazione per paragraph

% Definisci un nuovo comando per la subsubsubsection
% Definisci un nuovo comando per la subsubsubsection
\newcommand{\subsubsubsection}[1]{%
  \paragraph{#1}\mbox{}\\ % Usa la formattazione del paragraph e va a capo
  \addcontentsline{toc}{paragraph}{#1} % Aggiungi al sommario
}

