% !TEX encoding = UTF-8 Unicode
% !TEX TS-program = pdflatex

% toptesi document class
%\documentclass[%
%    a4paper, % not needed, by default it is a4paper, or also b5paper can be used
%    corpo=12pt, % dimension of basic font
%    % oneside is generally the way to go
%    oneside, % two side optimizes for two-face printing, having chapters open on the right (aka odd numbers), if you don't want blank pages put oneside here
%    stile=standard,
%    %evenboxes, % not needed, to put supervisors and candidate at the same level
%    tipotesi=magistrale,
%    numerazioneromana, % roman numbering for appendixes and preambles, up to Table of Contents
%    openright, % to force opening on the right for double-sided printing
%    cucitura=7mm, % for printing, 7mm should be enough
    %dvipsnames, % for compatibility with xcolor, it does not work
%]{toptesi}

\documentclass[11pt]{report}


\setcounter{secnumdepth}{4}
\setcounter{tocdepth}{4}

\input{common/packages}

% \setlength\textwidth{7in}
% \setlength\textheight{10in}
% \setlength\oddsidemargin{(\paperwidth-\textwidth)/2 - 1in}
% \setlength\topmargin{(\paperheight-\textheight
% -\headheight-\headsep-\footskip)/2 - 1in}

% this configuration is close to TopTesi in width
% \newgeometry{a4paper,top=3cm,bottom=3cm,left=3cm,right=3cm,%
% heightrounded}
% margins as for libreoffice writer
\newgeometry{top=2cm,bottom=2cm,left=2cm,right=2cm,%
heightrounded}


\lstdefinestyle{CStyle}{
    backgroundcolor=\color{backgroundColour},   
    commentstyle=\color{mGreen},
    keywordstyle=\color{magenta},
    numberstyle=\tiny\color{mGray},
    stringstyle=\color{mPurple},
    basicstyle=\footnotesize\ttfamily,
    breakatwhitespace=false,         
    breaklines=true,                 
    captionpos=b,                    
    keepspaces=true,                 
    numbers=left,                    
    numbersep=5pt,                  
    showspaces=false,                
    showstringspaces=false,
    showtabs=false,                  
    tabsize=2,
    language=C,
    lineskip=-0.1ex
}



% how to change Contents to Table of Contents
\addto\captionsenglish{% Replace "english" with the language you use
  \renewcommand{\contentsname}%
    {Table of Contents}%
}

% to change the name of Abbreviations to Acronyms
% not needed if use use entry types and define those
% \renewcommand{\abbreviationsname}{Acronyms}

% to allow line comments in algorithms
\algnewcommand{\LineComment}[1]{\State \(\triangleright\) #1}

% to declare abs and norm
\DeclarePairedDelimiter\abs{\lvert}{\rvert}%
\DeclarePairedDelimiter\norm{\lVert}{\rVert}%

% Swap the definition of \abs* and \norm*, so that \abs
% and \norm resizes the size of the brackets, and the 
% starred version does not.
\makeatletter
\let\oldabs\abs
\def\abs{\@ifstar{\oldabs}{\oldabs*}}
%
\let\oldnorm\norm
\def\norm{\@ifstar{\oldnorm}{\oldnorm*}}
\makeatother


\lstdefinestyle{CStyle}{
    backgroundcolor=\color{backgroundColour},   
    commentstyle=\color{mGreen},
    keywordstyle=\color{magenta},
    numberstyle=\tiny\color{mGray},
    stringstyle=\color{mPurple},
    basicstyle=\scriptsize\ttfamily,
    breakatwhitespace=false,         
    breaklines=true,                 
    captionpos=b,                    
    keepspaces=true,                 
    numbers=left,                    
    numbersep=5pt,                  
    showspaces=false,                
    showstringspaces=false,
    showtabs=false,                  
    tabsize=2,
    language=C,
    lineskip=-0.1ex
}

\newcommand{\Dedications}{%
    \chapter*{Dedications}
    \addcontentsline{toc}{chapter}{Dedications} % Aggiunge 'Dedications' all'indice
}

\newcommand{\paginavuota}{
    \newpage
    \thispagestyle{empty}
}

% Formato per subsubsection
\titleformat{\subsubsection}[block]
  {\normalfont\normalsize\bfseries} % Stile del titolo
  {\thesubsubsection} % Aggiungi il numero di subsubsection
  {1em} % Spaziatura tra numero e titolo
  {} % Comando dopo il titolo

\renewcommand\thesubsubsection{\thesection.\arabic{subsection}.\arabic{subsubsection}} % Formato della numerazione per subsubsection

% Formato per paragraph
\titleformat{\paragraph}[runin]
  {\normalfont\normalsize\bfseries} % Stile del titolo
  {\theparagraph} % Aggiungi il numero di paragraph
  {1em} % Spaziatura tra numero e titolo
  {} % Comando dopo il titolo

\renewcommand\theparagraph{\thesubsubsection.\arabic{paragraph}} % Formato della numerazione per paragraph

% Definisci un nuovo comando per la subsubsubsection
% Definisci un nuovo comando per la subsubsubsection
\newcommand{\subsubsubsection}[1]{%
  \paragraph{#1}\mbox{}\\ % Usa la formattazione del paragraph e va a capo
  \addcontentsline{toc}{paragraph}{#1} % Aggiungi al sommario
}



% change this configuration with your info
% if you need fewer or more supervisors you have to change \relatore command by adding or removing lines in the table in toptesi_config
\newcommand{\thesistitle}{Thesis TITLE}
\newcommand{\thesisuniversitylogo}{00_drafting/images/polito_logo_2021_blu} % choose your logo
\newcommand{\thesiscandidatename}{name}
\newcommand{\thesiscandidatesurname}{surname}
\newcommand{\thesissupervisoronetitle}{prof.}
\newcommand{\thesissupervisoronename}{name}
\newcommand{\thesissupervisoronesurname}{surname}
\newcommand{\thesissupervisortwotitle}{prof.}
\newcommand{\thesissupervisortwoname}{name}
\newcommand{\thesissupervisortwosurname}{surname}
\newcommand{\thesissupervisorthreetitle}{prof.}
\newcommand{\thesissupervisorthreename}{name}
\newcommand{\thesissupervisorthreesurname}{surname}
\newcommand{\thesissupervisorfourtitle}{prof.}
\newcommand{\thesissupervisorfourname}{name}
\newcommand{\thesissupervisorfoursurname}{surname}
\newcommand{\thesisdate}{DECEMBER 2024}
\newcommand{\thesiscourse}{COMPUTER ENGINEERING}
\newcommand{\thesisuniversity}{POLITECNICO DI TORINO}
\newcommand{\thesislevel}{MASTER} % master or bachelor
\newcommand{\thesiscandidatetext}{Candidate}
\newcommand{\thesissupervisortext}{Supervisors}


% fontsize is {size}{spacing}\family
\newcommand {\institutionfont}{\fontsize {22}{30}\scshape}
\newcommand {\divisionfont}{\fontsize {16}{20}\rmfamily}
\newcommand {\pretitlefont}{\fontsize {16}{16}\rmfamily}
\newcommand {\customtitlefont}{\fontsize {15}{18}\scshape}% {iwona}{bx}{n}}
\newcommand {\fixednamesfont}{\fontsize {12}{18}\mdseries}
\newcommand {\namesfont}{\fontsize {6.5}{11}\bfseries}
\newcommand {\candnamesfont}{\fontsize {10}{11}\bfseries}
\newcommand {\footfont}{\fontsize {15}{18}\rmfamily}

\setstretch{1.2}
%\onehalfspacing
%\doublespacing


\addbibresource{bibliography.bib}

% to load the glossaries, not needed if using bib2gls
\input{glossaries.tex}
\makeglossaries

\pagestyle{fancy}
\fancyhf{}  % Svuota le intestazioni esistenti
\fancyhead[C]{\itshape \leftmark}  % Titolo del capitolo in corsivo
\fancyfoot[C]{\thepage}  % Numero di pagina centrato in basso
% Personalizza \chaptermark per mostrare solo il titolo, senza "Chapter X"
\renewcommand{\chaptermark}[1]{\markboth{#1}{}}

\renewcommand{\thepage}{\Roman{page}}

\begin{document}

\input{common/config}

%\input{common/toptesi_config}

% front page
% frontespizio can be used for the first page print
% while the custom-made frontpage can be used as hard-cover
% use pdfjoin or pdfseparate to extract or put together the pages if needed
%\frontespizio* % without star the logo is on top
\input{common/frontpage.tex} % custom frontpage
%\retrofrontespizio
% insert text for the back of the front page
% if you insert any remove the following \paginavuota
% either a blank page or a back is needed to have double-sided printing
% pay attention to leave the space for the page

\paginavuota % clears a page

%\frontmatter

% abstract if needed
% \begin{abstract}
%     \input{content/abstract.tex}
% \end{abstract}

% to create blank pages for openright in frontmatter
% use one of the following two methods
% 1) use the following three lines
%\phantom{0} % needed otherwise cleardoublepage does not clean the page because it sees it empty
%\cleardoublepage
%\thispagestyle{empty} % to have empty page, without numbers
% 2) or
\paginavuota % to manually create a blank page

\pagenumbering{Roman}
%\sommario
\input{content/abstract}

\phantom{0}
\cleardoublepage
\thispagestyle{empty}

%\ringraziamenti
% acknowledgements

ACKNOWLEDGMENTS

\vspace*{5\baselineskip}

\begin{flushright}
    \textit{to...}
\end{flushright}


\tableofcontents


\listoffigures


\listoftables

% actually abbreviation is the name used for acronym in glossaries-extra
% title sets the name
% type tells the type of glossary to print
% style overrides the global style
% here we are printing only abbreviations
% printunsrtglossary if using record, otherwise printglossary is ok
\paginavuota
\printunsrtglossary[style=long,title=Acronyms,type=\acronymtype]

% also list of symbols here if needed

\input{common/post_summary_config.tex}

%\mainmatter

%\part{Prima Parte} % parts division, not needed
% Chapters always open on a right-side page, i.e. odd numbers, so a blank page is inserted if needed
%\cleardoublepage % to have a fully blank page
% a blank page appears before the first chapter in some configurations, on the last version it doesn't

\paginavuota
\pagenumbering{arabic}

% list here all the chapters
\chapter{Chapter One}

\section{Section name}
Lorem ipsum dolor sit amet, consectetur adipiscing elit, sed do eiusmod tempor incididunt ut labore et dolore magna aliqua. Ut enim ad minim veniam, quis nostrud exercitation ullamco laboris nisi ut aliquip ex ea commodo consequat. Duis aute irure dolor in reprehenderit in voluptate velit esse cillum dolore eu fugiat nulla pariatur. Excepteur sint occaecat cupidatat non proident, sunt in culpa qui officia deserunt mollit anim id est laborum. \cite{IEEEexample:article_typical}

\begin{figure}
    \centering
    \includegraphics[width=0.5\linewidth]{00_drafting/images/polito_logo_2021_blu_cut.jpg}
    \caption{Image Caption}
    \label{fig:enter-label}
\end{figure}


\begin{longtable}{ | m{4em} | m{4em} | m{4em} | m{4em} | m{4em} | m{4em} | m{4em} | }
\hline
\textbf{Col name} & \textbf{Col name} & \textbf{Col name} & \textbf{Col name} & \textbf{Col name} & \textbf{Col name} & \textbf{Col name} \\
\hline
\endfirsthead
\hline
\textbf{Col name} & \textbf{Col name} & \textbf{Col name} & \textbf{Col name} & \textbf{Col name} & \textbf{Col name} & \textbf{Col name} \\
\hline
\endhead

text & text & text & text & text & text & text \\
\hline


\caption{Long Table}
\label{tab:GPS_ICD} \\
\end{longtable}

\input{00_drafting/content/chapters/chap2}


% \paginavuota % it works even without stile=classica

\appendix
\input{content/appendixA}
\input{content/appendixB}

% endnotes here if needed

\phantom{0}
\cleardoublepage
\printbibliography[heading=bibintoc] % heading required to show it in ToC


\Dedications
\pagestyle{empty} 
\vspace*{4\baselineskip}

\vspace{\baselineskip}

Dedications for everyone

\end{document}
